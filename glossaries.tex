%%% Write your own glossary entries here
% The general structure is:
% \newglossaryentry{identifier}
% {
% 	name={name}, % Mandatory, what gets printed
% 	description={description of the entry}, % Mandatory, description that appears in the glossary index
% 	plural={plural-name}, % Optional, in case the plural is more complex
% 	sort={alphanumeric entry}, % Optional, how should the entry be sorted
% 	symbol={\ensuremath{associated symbol}}, % Optional, prints the symbol of the entry with \glssymbol{identifier}
% }

% Some examples
\newglossaryentry{test}
{
	name={test},
	description={This is a test entry for glossaries}
}
\newglossaryentry{timestepval}
{
	name={\ensuremath{x_t}},
	description={Value of variable $x$ at time step \ensuremath{t}}
}
\newglossaryentry{thermalstrain}
{
	name={\ensuremath{\epsilon^{th}}},
	description={Thermal strain component}
}
\newglossaryentry{plasticstrain}
{
	name={\ensuremath{\epsilon^{pl}}},
	description={Plastic strain component}
}
\newglossaryentry{inelasticstrain}
{
	name={\ensuremath{\epsilon^{in}}},
	description={Inelastic strain component}
}
\newglossaryentry{elasticstrain}
{
	name={\ensuremath{\epsilon^{el}}},
	description={Elastic strain component}
}
\newglossaryentry{totalstrain}
{
	name={\ensuremath{\epsilon^{tot}}},
	description={Total strain}
}
