\chapter{An overview of \LaTeX}

In this section, a few basic tools will be presented. In following sections more advance functionality and complex tools will be showcased. \textbf{Use this as examples and to your advantage!}

\section{Basics of \LaTeX}

\subsection{Text styles}

The following table showcases some of the more common text styles in \LaTeX.

\begin{table}[h]
	\centering
	\begin{tabular}{lcc}
	  \toprule
	  Style & Code & Ouput \\
	  \midrule
	  Quotes & \verb|``Quotes''| & ``Quotes'' \\
	  Boldface & \verb|\textbf{Boldface}| & \textbf{Boldface} \\
	  Italics & \verb|\textit{Italics}| & \textit{Italics} \\
	  Emphasis & \verb|\emph{Emphasis}| & \emph{Emphasis} \\
	  Underline & \verb|\underline{Underline}| & \underline{Underline} \\
	  Typewriter & \verb|\texttt{Typewriter}| & \texttt{Typewriter} \\
	  Mathematical & \verb|$Mathematical^{\pi\cdot i}$| & $Mathematical^{\pi\cdot i}$ \\
	  \LaTeX\ Comments & \verb|% Some text| & % Some text
	  \\
	  \bottomrule
	\end{tabular}
	\caption{Text styles in \LaTeX.}
	\label{fig:textstyles}
\end{table}

\subsection{Structure of a \LaTeX\ document}

For this template, which is based in the \texttt{book} class, we have the following major sections:

\begin{enumerate}
	\item \verb|\part{}|: Parts are fully self-contained portions of information. They leave a full blank page with only the title of the part. \textbf{This is not used and not recommended!}
	\item \verb|\chapter{}|: Your normal chapters, as you can see above. We are in the ``\emph{An overview of \LaTeX }.''
	\item \verb|\section{}|: Normal sections for a chapter. We are in ``\emph{Basics of \LaTeX}.''
	\item \verb|\subsection{}|: Subsections. We are in ``\emph{Structure of a \LaTeX\ document}.''
	\item \verb|\subsubsection{}|: Subsubsections. This level tends to be quite deep and will most likely not appear in the index unless we include \verb|\setcounter{secnumdepth}{3}| in the preamble\footnote{The preamble is the part before \texttt{\textbackslash begin\{document\}}, basically, the setup section.}.
	\item \verb|\paragraph{}|: One step deeper. By default paragraphs are not numbered.
\end{enumerate}

You jus have to write what you want between the \verb|{}| for each command, and \LaTeX\ does the rest. It typsets the titles/sections, it adds them to the table of contents and numbers them consistently!

\subsection{Mathematical notation}

\LaTeX\ provides several way to include symbols and write maths. The most basic way is to include mathematical notation or symbols into the text. This is known as \emph{inline} and can be done with \verb|$...$|. Whatever is between the \$ symbols, is typeset in mathematical notation. This is an example: $2 = \frac{4}{2}$. This is produced using \verb|$2 = \frac{4}{2}$|.

Another method is to write mathematical formulas in \emph{display} mode, which is separated from the text. This can be done by wrapping the text in \verb|\[...\]|. \textbf{This is not recommended} as the next method is better. Here is an example:

\[
	2 = \frac{4}{2}
\]

Normally, the best way is to use mathematical environments. This environments will provide more functionality and generally number the equations and allows them to be labelled. Here are a few examples:

\begin{equation} \label{eq:simpleeq}
	2 = \frac{4}{2}
\end{equation}

The equation above, \cref{eq:simpleeq}, is produced by writing:

\begin{lstlisting}[language={[LaTeX]TeX}]
\begin{equation} \label{eq:simpleeq}
	2 = \frac{4}{2}
\end{equation}
\end{lstlisting}

Lets showcase some more environments that help us write beautiful formulas! The \verb|\begin{array}| environment helps us write vertically aligned formulas!

\begin{equation} \label{eq:abaqus-exponential-decay}
f(t) = \left\{
	\begin{array}{lcc}
		A_0 + A\cdot e^{-\dfrac{t - t_0}{t_d}} & for & t \geq t_0 \\
		A_0 & for & t < t_0
	\end{array}
	\right.
\end{equation}

\begin{lstlisting}[language={[LaTeX]TeX}]
\begin{equation} \label{eq:abaqus-exponential-decay}
	f(t) = \left\{
	\begin{array}{lcc}
		A_0 + A\cdot e^{-\dfrac{t - t_0}{t_d}} & for & t \geq t_0 \\
		A_0 & for & t < t_0
	\end{array}
	\right.
\end{equation}
\end{lstlisting}


The \verb|\begin{aling}| environment may be easier to use, but it has a few quirks. Read the documentation\footnote{\url{http://tug.ctan.org/info/short-math-guide/short-math-guide.pdf}} for more information.

\begin{align}
	a_{11}& =b_{11}&
	a_{12}& =b_{12}\\
	a_{21}& =b_{21}&
	a_{22}& =b_{22}+c_{22}
\end{align}

\begin{lstlisting}[language={[LaTeX]TeX}]
\begin{align}
	a_{11}& =b_{11}&
	a_{12}& =b_{12}\\
	a_{21}& =b_{21}&
	a_{22}& =b_{22}+c_{22}
\end{align}
\end{lstlisting}

The \verb|\begin{subequations}| allows us to have several formulas numbered into the same reference. As shown in \cref{eq:symmetry-bc}, with the first entry being \cref{eq:x-symmetry-bc}.

\begin{subequations} \label{eq:symmetry-bc}
	\begin{equation} \label{eq:x-symmetry-bc}
		\text{\texttt{XSYMM}} \equiv U1 = UR2 = UR3 = 0
	\end{equation}
	\begin{equation}
		\text{\texttt{ZSYMM}} \equiv U3 = UR1 = UR2 = 0
	\end{equation}
\end{subequations}

\begin{lstlisting}[language={[LaTeX]TeX}]
\begin{subequations} \label{eq:symmetry-bc}
	\begin{equation} \label{eq:x-symmetry-bc}
		\text{\texttt{XSYMM}} \equiv U1 = UR2 = UR3 = 0
	\end{equation}
	\begin{equation}
		\text{\texttt{ZSYMM}} \equiv U3 = UR1 = UR2 = 0
	\end{equation}
\end{subequations}
\end{lstlisting}

\subsection{References}

One of the strongest points of \LaTeX\ is its wonderful and powerful referencing system. We can reference whatever we want by putting on a ``tag'' with the command \verb|\label{xxx}|. Wherever the \verb|\label| is, it will refer to it. You can see some examples above where we refered to a few equations by their labels, which are inside the \verb|\begin{equation}| environment. This way, \LaTeX\ knows automatically what type of thing they are referring.

Here, lets see what types of refences we can generate!

\begin{table}[h]
	\centering
	\begin{tabular}{lcc}
	  \toprule
	  Package & Command & Result \\
	  \midrule
	  \LaTeX & \verb|\ref{eq:simpleeq}| & \ref{eq:simpleeq} \\
	  & \verb|\pageref{eq:simpleeq}| & \pageref{eq:simpleeq} \\
	  \cmidrule{2-3}
	  \texttt{hyperref} & \verb|\autoref{eq:simpleeq}| & \autoref{eq:simpleeq} \\
			  & \verb|\autoref{fig:textstyles}| & \autoref{fig:textstyles} \\
			  & \verb|\autopageref{eq:simpleeq}| & \autopageref{eq:simpleeq} \\
	  \cmidrule{2-3}
	  \texttt{cleveref} & \verb|\cref{eq:simpleeq}| & \cref{eq:simpleeq} \\
			  & \verb|\Cref{eq:simpleeq}| & \Cref{eq:simpleeq} \\
	  & \verb|\cpageref{eq:simpleeq}| & \cpageref{eq:simpleeq} \\
			  & \verb|\cref{eq:simpleeq,eq:symmetry-bc}| & \cref{eq:simpleeq,eq:symmetry-bc} \\
			  & \verb|\crefrange{eq:simpleeq}| & \multirow{2}{*}{\crefrange{eq:simpleeq}{eq:symmetry-bc}}\\
	  & \verb|{eq:symmetry-bc}| & \\
	  \bottomrule
	\end{tabular}
	\caption[Different reference mechanisms.]{Different reference mechanisms. \textbf{The author recommends \texttt{cleveref}!}. It is included in this template.}
	\label{tab:reference-systems}
\end{table}

\subsection{Bibliography}

Bibliography management is another strong point of \LaTeX! We just need to add bibliographic entries to the bibliography database, which for this template it is the \texttt{main.bib} file. Here is what such an entry can look like:

\begin{lstlisting}[language={[LaTeX]TeX}]
@book{lovecraft2016el,
	author = {Lovecraft, H. P.},
	title = {El clérigo malvado y otros relatos},
	publisher = {Alianza Editorial},
	year = {2016},
	address = {Madrid},
	isbn = {9788491042105}
}
\end{lstlisting}

In order to cite the entry we just have to use \verb|\cite{}| with the entry's identifier, like so \verb|\cite{lovecraft2016el}| \cite{lovecraft2016el}. We can also have multiple cites in the same command, \cite{lovecraft2016el,norton_creep} (\verb|\cite{lovecraft2016el,norton_creep}|). It is that simple! They get automatically printed in the bibliography section.

\textsc{\color{red}Important:} this template uses \texttt{biblatex} as the management system, which is a powerful, flexible and modern tool. Therefore, you will need to run the \texttt{biber} command to build the bibliography after the first compilation of your document; then you will have to recompile after \texttt{biber} has run. Most editors do this by default.

\subsection{Tables, images and floating environments}

Probably, the part of \LaTeX\ that causes the most confusion among new users, are the so called \emph{floating envrionments}. \textbf{Tables, images, algorithms, etc are floating envrionments}. This means that \textbf{\LaTeX\ can position them where it sees fit, not where they are written by the user.} In reality, \LaTeX\ is trying to optimise your document's layout and leave as little empty space as possible.

Sooo... How do we solve \LaTeX\ moving our floating environments? Here are a few solutions:

\begin{itemize}
	\item We don't solve it. \LaTeX\ referencing tools allow us to easily point the reader to the table, image, etc. Therefore, it is not that problematic that the \emph{floats} may not be where we put them!
	\item We can ask \LaTeX\ to try to place the image where it appears in our document. This is done with the \emph{``here''} \verb|[h]| placement modifier, more on placement modifiers later. \textbf{This is not a definitive solution.} This will just tell \LaTeX\ to try hard to do what we are asking. There is the \verb|[h!]| modifier, which is even stronger.
	\item \textbf{A really good solution is to use} \verb|\FloatBarrier|. It comes from the \verb|placeins| package, included in this template. \verb|\FloatBarrier| forces \LaTeX\ to put all floating environment that have already appear before the position where \verb|\FloatBarrier| appears. This is very useful to force \LaTeX\ to put all floats before another section that may not be related to the topic of those floats. Here is an example:

\begin{lstlisting}[language={[LaTeX]TeX}]
\section{Some topic}

\begin{figure}
	XXX
\end{figure}

\begin{table}
	XXX
\end{table}

\FloatBarrier % All previous floats will appear before this point.

\section{Some unrelated topic}
XXX
\end{lstlisting}
	\item We can use the placement modifier \verb|[H]| to force the float to appear \emph{HERE}. This is provided by the \verb|float| package. However, \textbf{this solution is not recommended!} It can lead to some wierd and nasty document layouts!
\end{itemize}

Now, how do we actually include figures, tables, etc? They all follow the same structure, here are some examples:

\begin{description}
		\item[Figures] are declared in the \verb|figure| environment (\emph{\textsc{shock!}}). You can see the image rendered in \cref{fig:monoblock-overview-mesh}.

\begin{lstlisting}[language={[LaTeX]TeX}]
\begin{figure}
	\centering % Center image horizontally
	\includegraphics[keepaspectratio, trim = 1050 12 150 30, clip, width=0.5\linewidth, height=0.3\textheight]{Images/monoblock-material-overview-mesh.png}
	\caption[Overview of \glsentryname{FEM} mesh used for the final analysis.]{Overview of \glsxtrshort{FEM} mesh used for the final analysis.}
	\label{fig:monoblock-overview-mesh}
\end{figure}
\end{lstlisting}
	The key here is \verb|\includegraphics|, it is what loads the graphics and allows us to set its properties. The above example is rather complex, most times you do not need these many options. Nonetheless, here is what they do:
	\begin{description}
			\item[\texttt{keepaspectratio}] Keeps the size ratio of the image. Very useful if you set \verb|height| and \verb|width| at the same time.
			\item[\texttt{trim}] It allows us to trim/cut the image. It cuts X amount of pixels from the \texttt{left, bottom, right, top}. This is useful if your image is too large and you only care about a small portion of it.
			\item[\texttt{clip}] Only show the trimmed image.
			\item[\texttt{width} and \texttt{height}] Sets the maximum size with respect to the width and height. We use \verb|\linewidth| and \verb|\textheight| to limit the size of the image in the page by using the page's natural lengths.
	\end{description}

	Then we have \verb|\caption|, which is what adds the text to the image. We use \verb|\caption[]| here, to modify the text that will appear in the ``List of Figures'', as I do not want my acronym \texttt{FEM} to be linked there, and therefore I use \verb|\glsentryname| to control that. But more about acronyms and glossaries in \cref{sec:glossaries}
	Finally, we have \verb|\label{}| is is what allows us to give an identifier to our image so that we can reference it.
		\item[Tables] are fairly easy to do once we get used to their nature. It uses the \verb|table| floating environment with another environment that allows us to type tabulated data. A basic example is given below and shown in \cref{tab:example-table}.
\begin{lstlisting}[language={[LaTeX]TeX}]
\begin{table}
	\centering % To center the table
	\begin{tabular}{lcr}
	  \toprule
	  Heading 1 & Heading 2 & Heading 3 \\
	  \midrule
	  Left aligned & Center aligned & Right aligned \\
	  \cmidrule{2-3} % Example of a controled rule
	  Some info & Some info & Some info \\
	  \bottomrule
	\end{tabular}
	\caption{Example of a table.}
	\label{tab:example-table}
\end{table}
\end{lstlisting}
	\textsc{\color{red}Important:} the \texttt{\&} symbol is used as a column break in all alignment environments!

	The column alingment options for the tabular environment can be \texttt{l, c, r, m{}} or \texttt{p{}} among others. They refer to left, center, right alignment and the \texttt{m{}} and \texttt{p{}} refer to a limited size column whose vertical alignment is either centered or natural. You could use them as \verb|m{0.3\linewidth}| for example. If you want to aling the text horizontally with \texttt{m} or \texttt{p} you would write \verb|>{\centering\arraybackslash}m{0.3\linewidth}|. You can change the \verb|\centering| for a \verb|\raggedright| for a right aligned column. But this is getting too advance!
	One final bit of knowledge about very long tables and dynamicly sized columns. This template includes the package \texttt{xltabular}, which includes the well-known \verb|tabularx| environment and merges it with the \verb|longtable| environment (for tables that can span more than one page) generating its own \verb|xltabular| envrionment. Please red the documentation of the \texttt{tabularx}, \verb|longtable| and \verb|xltabular| if you need to build complex tables!

	Also, \textbf{there are online tools to help you generate \LaTeX\ tables from Excel sheets}. One such example (which I am not very familiar with nor endorse) is \href{https://tableconvert.com/excel-to-latex}{Tableconvert}.

	Finally, if you have \texttt{.csv} files or similar and you want to print them in your \LaTeX\ document, you can see \cref{tab:automatic-reading-csv}, which was automatically generated using \verb|pgfplotstable|.
\end{description}

\begin{figure}[h]
	\centering % Center image horizontally
	\includegraphics[keepaspectratio, trim = 1050 12 150 30, clip, width=0.5\linewidth, height=0.3\textheight]{Images/monoblock-material-overview-mesh.png}
	\caption[Overview of \glsentryname{FEM} mesh used for the final analysis.]{Overview of \glsxtrshort{FEM} mesh used for the final analysis.}
	\label{fig:monoblock-overview-mesh}
\end{figure}

\begin{table}[h]
	\centering % To center the table
	\begin{tabular}{lcr}
	  \toprule
	  Heading 1 & Heading 2 & Heading 3 \\
	  \midrule
	  Left aligned & Center aligned & Right aligned \\
	  \cmidrule{2-3} % Example of a controled rule
	  Some info & Some info & Some info \\
	  \bottomrule
	\end{tabular}
	\caption{Example of a table.}
	\label{tab:example-table}
\end{table}

\section{Glossaries}\label{sec:glossaries}

\section{Creating beautiful plots in 2D and 3D}

\section{Automatic formatting of table data}

The following table, \cref{tab:automatic-reading-csv}, is formatted using the following general setup for \verb|pgfplotstable|. This is only needed once, an it applies to ``all'' the automatically loaded table.

\begin{lstlisting}[language={[LaTeX]TeX}]
% Configure the general setting of pgfplotstable
\pgfplotstableset{
	every odd row/.style={
		before row={\rowcolor{gray!20}}
	},
	every head row/.style={
		before row=\toprule,
		after row=\midrule,
		% Don't print the row name or the row index!
		output empty row
	},
	every last row/.style={
		after row=\bottomrule
	},
	header=false,
	format=file,
	col sep=tab,
	search path={Data},
	font={\small}
}
\end{lstlisting}

And then the actual table:
\begin{lstlisting}[language={[LaTeX]TeX}]
\begin{table}
	\newcommand{\prop}{Expansion}
	\newcommand{\propunit}{[\unit{\milli\meter\per\celsius\per\milli\meter}]}
	\centering
	\pgfplotstabletypeset[
	every head row/.append style={
		before row={
			\toprule
			\multicolumn{2}{c}{\glsentryname{Cu-OFHC}} \\
			\midrule
			\multirow{2}{\widthof{\propunit}}{\centering \prop\ \propunit} & \multirow{2}{\widthof{Temperature}}{\centering Temperature [\unit{\celsius}]} \\
			\\
		},
	},
	]{ITER Cu You-harden for WPDIV phase II_\prop_f_T.txt}
	\caption{Automatically formatted table using \texttt{pgfplotstable}.}
	\label{tab:automatic-reading-csv}
\end{table}
\end{lstlisting}

Whats even cooler is that \verb|pgfplotstable| uses the package \verb|siunitx| to format the values as it is included in this template!

% Configure the general setting of pgfplotstable
\pgfplotstableset{
	every odd row/.style={
		before row={\rowcolor{gray!20}}
	},
	every head row/.style={
		before row=\toprule,
		after row=\midrule,
		% Don't print the row name or the row index!
		output empty row
	},
	every last row/.style={
		after row=\bottomrule
	},
	header=false,
	format=file,
	col sep=tab,
	search path={Data},
	font={\small}
}

\begin{table}[h]
	\newcommand{\prop}{Expansion}
	\newcommand{\propunit}{[\unit{\milli\meter\per\celsius\per\milli\meter}]}
	\centering
	\pgfplotstabletypeset[
	every head row/.append style={
		before row={
			\toprule
			\multicolumn{2}{c}{\glsentryname{Cu-OFHC}} \\
			\midrule
			\multirow{2}{\widthof{\propunit}}{\centering \prop\ \propunit} & \multirow{2}{\widthof{Temperature}}{\centering Temperature [\unit{\celsius}]} \\
			\\
		},
	},
	]{ITER Cu You-harden for WPDIV phase II_\prop_f_T.txt}
	\caption{Automatically formatted table using \texttt{pgfplotstable}.}
	\label{tab:automatic-reading-csv}
\end{table}


% Emacs setup

%%% Local Variables:
%%% mode: latex
%%% coding: utf-8
%%% TeX-engine: luatex
%%% TeX-master: "../report"
%%% End:
