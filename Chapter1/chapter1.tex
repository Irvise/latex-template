\chapter{An overview of \LaTeX}

In this section, a few basic tools will be presented. In following sections more advance functionality and complex tools will be showcased. Use this to your advantage!

\section{Basics of \LaTeX}

\subsection{Text styles}

The following table showcases some of the more common text styles in \LaTeX.

\begin{table}[h]
	\centering
	\begin{tabular}{lcc}
	  \toprule
	  Style & Code & Ouput \\
	  \midrule
	  Quotes & \verb|``Quotes''| & ``Quotes'' \\
	  Boldface & \verb|\textbf{Boldface}| & \textbf{Boldface} \\
	  Italics & \verb|\textit{Italics}| & \textit{Italics} \\
	  Emphasis & \verb|\emph{Emphasis}| & \emph{Emphasis} \\
	  Underline & \verb|\underline{Underline}| & \underline{Underline} \\
	  Typewriter & \verb|\texttt{Typewriter}| & \texttt{Typewriter} \\
	  Mathematical & \verb|$Mathematical^{\pi\cdot i}$| & $Mathematical^{\pi\cdot i}$ \\
	  \bottomrule
	\end{tabular}
	\caption{Text styles in \LaTeX.}
	\label{fig:textstyles}
\end{table}

\subsection{Structure of a \LaTeX\ document}

For this template, which is based in the \texttt{book} class, we have the following major sections:

\begin{enumerate}
	\item \verb|\part{}|: Parts are fully self-contained portions of information. They leave a full blank page with only the title of the part. \textbf{This is not used and not recommended!}
	\item \verb|\chapter{}|: Your normal chapters, as you can see above. We are in the ``\emph{An overview of \LaTeX }.''
	\item \verb|\section{}|: Normal sections for a chapter. We are in ``\emph{Basics of \LaTeX}.''
	\item \verb|\subsection{}|: Subsections. We are in ``\emph{Structure of a \LaTeX\ document}.''
	\item \verb|\subsubsection{}|: Subsubsections. This level tends to be quite deep and will most likely not appear in the index unless we include \verb|\setcounter{secnumdepth}{3}| in the preamble\footnote{The preamble is the part before \texttt{\textbackslash begin\{document\}}, basically, the setup section.}.
	\item \verb|\paragraph{}|: One step deeper. By default paragraphs are not numbered.
\end{enumerate}

You jus have to write what you want between the \verb|{}| for each command, and \LaTeX\ does the rest. It typsets the titles/sections, it adds them to the table of contents and numbers them consistently!

\subsection{Mathematical notation}

\LaTeX\ provides several way to include symbols and write maths. The most basic way is to include mathematical notation or symbols into the text. This is known as \emph{inline} and can be done with \verb|$...$|. Whatever is between the \$ symbols, is typeset in mathematical notation. This is an example: $2 = \frac{4}{2}$. This is produced using \verb|$2 = \frac{4}{2}$|.

Another method is to write mathematical formulas in \emph{display} mode, which is separated from the text. This can be done by wrapping the text in \verb|\[...\]|. \textbf{This is not recommended} as the next method is better. Here is an example:

\[
	2 = \frac{4}{2}
\]

Normally, the best way is to use mathematical environments. This environments will provide more functionality and generally number the equations and allows them to be labelled. Here are a few examples:

\begin{equation} \label{eq:simpleeq}
	2 = \frac{4}{2}
\end{equation}

The equation above, \cref{eq:simpleeq}, is produced by writing:

\begin{lstlisting}[language={[LaTeX]TeX}]
\begin{equation} \label{eq:simpleeq}
	2 = \frac{4}{2}
\end{equation}
\end{lstlisting}

Lets showcase some more environments that help us write beautiful formulas! The \verb|\begin{array}| environment helps us write vertically aligned formulas!

\begin{equation} \label{eq:abaqus-exponential-decay}
f(t) = \left\{
	\begin{array}{lcc}
		A_0 + A\cdot e^{-\dfrac{t - t_0}{t_d}} & for & t \geq t_0 \\
		A_0 & for & t < t_0
	\end{array}
	\right.
\end{equation}

\begin{lstlisting}[language={[LaTeX]TeX}]
\begin{equation} \label{eq:abaqus-exponential-decay}
	f(t) = \left\{
	\begin{array}{lcc}
		A_0 + A\cdot e^{-\dfrac{t - t_0}{t_d}} & for & t \geq t_0 \\
		A_0 & for & t < t_0
	\end{array}
	\right.
\end{equation}
\end{lstlisting}


The \verb|\begin{aling}| environment may be easier to use, but it has a few quirks. Read the documentation\footnote{\url{http://tug.ctan.org/info/short-math-guide/short-math-guide.pdf}} for more information.

\begin{align}
	a_{11}& =b_{11}&
	a_{12}& =b_{12}\\
	a_{21}& =b_{21}&
	a_{22}& =b_{22}+c_{22}
\end{align}

\begin{lstlisting}[language={[LaTeX]TeX}]
\begin{align}
	a_{11}& =b_{11}&
	a_{12}& =b_{12}\\
	a_{21}& =b_{21}&
	a_{22}& =b_{22}+c_{22}
\end{align}
\end{lstlisting}

The \verb|\begin{subequations}| allows us to have several formulas numbered into the same reference. As shown in \cref{eq:symmetry-bc}, with the first entry being \cref{eq:x-symmetry-bc}.

\begin{subequations} \label{eq:symmetry-bc}
	\begin{equation} \label{eq:x-symmetry-bc}
		\text{\texttt{XSYMM}} \equiv U1 = UR2 = UR3 = 0
	\end{equation}
	\begin{equation}
		\text{\texttt{ZSYMM}} \equiv U3 = UR1 = UR2 = 0
	\end{equation}
\end{subequations}

\begin{lstlisting}[language={[LaTeX]TeX}]
\begin{subequations} \label{eq:symmetry-bc}
	\begin{equation} \label{eq:x-symmetry-bc}
		\text{\texttt{XSYMM}} \equiv U1 = UR2 = UR3 = 0
	\end{equation}
	\begin{equation}
		\text{\texttt{ZSYMM}} \equiv U3 = UR1 = UR2 = 0
	\end{equation}
\end{subequations}
\end{lstlisting}

\subsection{References}

One of the strongest points of \LaTeX\ is its wonderful and powerful referencing system. We can reference whatever we want by putting on a ``tag'' with the command \verb|\label{xxx}|. Wherever the \verb|\label| is, it will refer to it. You can see some examples above where we refered to a few equations by their labels, which are inside the \verb|\begin{equation}| environment. This way, \LaTeX\ knows automatically what type of thing they are referring.

Here, lets see what types of refences we can generate!

\begin{table}[h]
	\centering
	\begin{tabular}{lcc}
	  \toprule
	  Package & Command & Result \\
	  \midrule
	  \LaTeX & \verb|\ref{eq:simpleeq}| & \ref{eq:simpleeq} \\
	  & \verb|\pageref{eq:simpleeq}| & \pageref{eq:simpleeq} \\
	  \cmidrule{2-3}
	  \texttt{hyperref} & \verb|\autoref{eq:simpleeq}| & \autoref{eq:simpleeq} \\
			  & \verb|\autoref{fig:textstyles}| & \autoref{fig:textstyles} \\
			  & \verb|\autopageref{eq:simpleeq}| & \autopageref{eq:simpleeq} \\
	  \cmidrule{2-3}
	  \texttt{cleveref} & \verb|\cref{eq:simpleeq}| & \cref{eq:simpleeq} \\
			  & \verb|\Cref{eq:simpleeq}| & \Cref{eq:simpleeq} \\
	  & \verb|\cpageref{eq:simpleeq}| & \cpageref{eq:simpleeq} \\
			  & \verb|\cref{eq:simpleeq,eq:symmetry-bc}| & \cref{eq:simpleeq,eq:symmetry-bc} \\
			  & \verb|\crefrange{eq:simpleeq}| & \multirow{2}{*}{\crefrange{eq:simpleeq}{eq:symmetry-bc}}\\
	  & \verb|{eq:symmetry-bc}| & \\
	  \bottomrule
	\end{tabular}
	\caption[Different reference mechanisms.]{Different reference mechanisms. \textbf{The author recommends \texttt{cleveref}!}. It is included in this template.}
	\label{tab:reference-systems}
\end{table}

\subsection{Bibliography}

Bibliography management is another strong point of \LaTeX!


% Emacs setup

%%% Local Variables:
%%% mode: latex
%%% coding: utf-8
%%% TeX-engine: luatex
%%% TeX-master: "../report"
%%% End:
